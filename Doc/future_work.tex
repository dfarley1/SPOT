\subsection{Hardware}
The hardware was primarily chosen based on its ease of use and quick prototyping for quick implementation with the database.
We were focused on getting a working prototype to make sure that our idea was possible. 
We've realized that we really only need a CPU, GPU, Bluetooth transceiver, Wifi transceiver, camera, and proximity sensor. 
Using only these components we could develop our own PCB that is an optimized version of our prototype. 

It would also be beneficial if you didn't have to do a serial connection between the raspberry pi and the arduino nano. There is about a 2 second delay for the light to turn green or red which isn't the worst thing ever however it is still a little bit of an annoyance. 

The Camera software can be signficantly improved with additional image preprocessing. Currently we directly hand our images to our openALPR command. This only yields about 80\%-90\% depending on the angle. We could improve it by training the openALPR data with our own images.

\subsubsection{Power Efficient Device}
Cutting down on hardware, we can foresee a significant drop it power consumption. 
The Raspberry Pi Model 3 itself is extremely overkill and we believe that it can be marked down significantly.
Also there is an extreme amount of power that is used to power the neopixel rings. It may be best to use something else for the status light. 
For instance, we do not need all 40 GPIO pins, multi-core processors, or even all the USB ports.
Additionally, power-conserving software can be implemented to ensure our system is only at full power when absolutely necessary and remains idle or in a very low powered state normally.

\subsubsection{Magnetometer}
Currently, our ping sensor relies on sound waves. 
This requires that there is not high acoustic impedance, meaning we can't fully enclose the proximity sensor into our package. 
As discussed in the sensor section of this paper, the ping sensor relies heavily on a flat object.
For car bumpers this is not ideal since most cars are curved.
This is also undesirable since we would ideally like to weather-proof our whole design to design for longevity. 
Using other sensors, like a magnetometer, can provide a different means of sensing that can also allow for full weather-proofing of the device.
Magnetometers simplify our device to: 
'If there is a chunk of metal in front of me, signal high. 
Otherwise leave the GPIO pin low.'

\subsection{Mechanical Design}
Currently the mechanical design isn't as sturdy as we would have hoped.
We would like to one day incorporate a LCD screen into the module and find a tube that doesn't shatter as easily as acrylic. 
Also the magnets are one of the most expensive parts of the hardware. 
If possible I would recommend redesigning without them. 
It might be beneficial to also create a different sensor module that could be produced and assembled faster.
It takes approximately one full week to print all the necessary parts. 
It would also be handy to create a low cost module that doesn't have a camera but can easily detect whether or not a car is there. 
It would be nice to actually make several modules, ranging in various prices, in order to meet the needs of the potential customer.
Experimenting with a Raspberry Pi 0W would be another option too. 

\subsection{Cloud}
\begin{itemize}
    \item Refactor the sensor API to utilize DRF and JSON transactions
    \item Complete the implementation of payment methods
    \item Redesign the payment calculation algorithm as described above.
    \item Implement license plate verification
    \item Add the ability for SPOTs, sections, and lots to all define their own rates with the ability to fallthrough to the containing region
    \item Define a manager account and add checks on edit pages to make sure the user is a manager
\end{itemize}  